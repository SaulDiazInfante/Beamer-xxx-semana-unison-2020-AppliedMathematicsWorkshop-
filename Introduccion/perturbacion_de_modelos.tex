\begin{frame}
  \frametitle{Por que EDEs?}
	\begin{empheq}[box={\Garybox[En ocaciones]}]{align*}
		EDO+ruido=Mejor \text{ } modelo
	\end{empheq}
	\begin{overlayarea}{\textwidth}{.8\textheight}
		\begin{columns}
    		\column{.55\textwidth}
			\only<2-3>{
			\begin{exampleblock}{Crecimiento de Poblaciones}
				$$
					\frac{dN}{dt}=a(t)N(t) \qquad N_0=N(0)=cte.
				$$
			\end{exampleblock}
			}
			\only<4-8>{
			\begin{exampleblock}{Circuitos Eléctricos}
			\begin{align*}
				&L\cdot Q''(t)+
				R\cdot Q'(t)+
				\frac{1}{C}\cdot Q(t)
				=F(t)\\
				&Q(0)=Q_0\\
				&Q'(0)=I_0
			\end{align*}
		\end{exampleblock}
		}
		\only<7-8>{
		\begin{empheq}[box=\shadowbox*]{equation*}
			Q(t)=Z(t)+"ruido"
		\end{empheq}
		}
		\column{.58\textwidth}
		\only<3>{
			\begin{empheq}[box=\shadowbox*]{equation*}
				a(t)=r(t)+"ruido"
			\end{empheq}
		}
		\only<5-8>{
			%\includegraphics[width=\textwidth]{./images/CircuitRLC.png}
	     \begin{circuitikz}[american voltages]
	      \draw (0,0)
      		to[sV,v=$F(t)$] (0,2) % The voltage source
      		to[R=$R$, i^>=$i(t)$] (2,2) % The resistor
          to[L=$L$] (4,2)
          to[C=$C$] (4,0)
          --(0,0) ;
			\end{circuitikz}
		}
		\only<6-8>{
		\begin{empheq}[box=\shadowbox*]{equation*}
			F(t)=G(t)+"ruido"
		\end{empheq}
		}
		\only<8>{
			Estima $Z(t)$ observando $Q(t)$
		}
	\end{columns}
	\end{overlayarea}
\end{frame}
%%%%%%%%%%%%%%%%%%%%%%%%%%%%%%%%%%%%%%%%%%%%%%%%%%%%%%%%%%%%%%%%%%%%%%%%%%%%%%%%%
\begin{frame}
  \frametitle{Para fijar ideas}
  \begin{empheq}[box={\Garybox[Ejemplo]}]{align*}
 		dN(t) = aN(t)dt
 	\end{empheq}
   \begin{overlayarea}{\textwidth}{.3\textheight}
     \begin{columns}
       \column{.5\textwidth}
        \only<2->{
          \begin{block}{Perturba sobre $[t, t+dt)$}
            \only<3->{
            $$
              a dt
              \rightsquigarrow
              a dt + \sigma dB(t)
            $$
            }
           \end{block}
		       }
        \column{.5\textwidth}
          \only<4->{
            \begin{exampleblock}{obten una EDE}
              $$
               dN(t) = aN(t)dt + \sigma N(t) dB(t)
              $$
            \end{exampleblock}
            }
     \end{columns}
   \end{overlayarea}
  \begin{overlayarea}{\textwidth}{.7\textheight}
    \centering
    \resizebox{0.45\textwidth}{!}{%
     \only<5>{
     \begin{tikzpicture}
       \begin{axis}[%
         line width=1.0pt,
         mark size=1.0pt
         ]%
         \addplot [color=blue]%
         table [%
           x index = {0},
           y index = {1}
           ]{\mydata};
           \addplot[domain=0:5, samples=100]{1.5*exp(x)};
       \end{axis}
     \end{tikzpicture}
     }
  }
  \end{overlayarea}
\end{frame}
%%%%%%%%%%%%%%%%%%%%%%%%%%%%%%%%%%%%%%%%%%%%%%%
 \begin{frame}
 	\frametitle{¿Por que hacer métodos numéricos para EDEs?}
 	\begin{empheq}[box={\Garybox[En ocaciones]}]{align*}
 		EDO+ruido=Mejor \text{ } modelo
 	\end{empheq}
 	\begin{overlayarea}{\textwidth}{.7\textheight}
 		\begin{columns}
     		\column{.5\textwidth}
 				\begin{alertblock}{Solución analítica?}
 					muy RARA
 				\end{alertblock}
 			\column{.5\textwidth}
 			\only<2>{
 				\begin{block}{Usa }
 					Teoría de diferencias finitas y haz una extención estocástica.
 				\end{block}
 			}
 		\end{columns}
 	\end{overlayarea}
 \end{frame}
 %%%%%%%%%%%%%%%%%%%%%%%%%%%%%%%%%%%%%%%%%%%%%%%%
 \begin{frame}
 	\frametitle{Objetivo}
 	\begin{alertblock}{Objetivo de la charla}
 		\textbf{Ilustrar} como aproximar soluciones de EDEs  a partir de
 \emph{conocimientos básicos}
 		de los \textbf{métodos deterministas} y nociones muy elementales de
variables 		 aleatorias.
 	\end{alertblock}
 \end{frame}
%%%%%%%%%%%%%%%%%%%%%%%%%%%%%%%%%%%%%%%%%%%%%%%%%%%%%%%%%%%%%%%%%%%%%%%%%%%%%%
