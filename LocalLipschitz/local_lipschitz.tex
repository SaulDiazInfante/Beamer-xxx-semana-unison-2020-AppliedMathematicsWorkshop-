\begin{frame}[plain]
	%\frametitle{EM diverge en sentido d\'ebil y fuerte.}    
	\vspace{-1cm}
		\begin{overlayarea}{\textwidth}{\textheight}
			\begin{columns}
				\column[t]{.3\textwidth}
					\textcolor{cyan}{Ejemplo:}
					\scalebox{.7}{\parbox{.5\linewidth}{%
					\begin{align*}
						dy(t) &= -10 \sign(y(t))|y(t)|^{\num{1.1}} dt + 4dW_t, \\
						y_0 &= 0, \quad t\in [0,10] \\
						&\approx \EX{|y(10)|}, \quad \num{e4}\text{ trayectorias }, \\
						& h=10/N, \quad N=\{1, 2,\dots,  50\}
					\end{align*}
					}
				}
				\column[t]{.6\textwidth}
					\only<2->{
						\vspace*{.12cm}
						\includegraphics[width=\textwidth]%
						{./IMAGENES/LIPSCHITZ/HutzenthalerExperiment.png}
					}
		\end{columns}
		%\vspace{0.3cm}
		\only<3->{
			\begin{bibunit}[alpha]
				\nocite{Hutzenthaler2010}
				\putbib
			\end{bibunit}
		}
		\end{overlayarea}
	\end{frame}
%%%%%%%%%%%%%%%%%%%%%%%%%%%%%%%%%%%%%%%%%%%%%%%%%%%%%%%%%%%%%%%%%%%%%
	\begin{frame}[plain]
	\frametitle{Modelos con Condiciones Local Lipschitz}
	\begin{columns}
		\column{.3\textwidth}
			\begin{overlayarea}{\textwidth}{.3\textheight}
			  \begin{itemize}[<+-|alert@+>] 	
				  \item 
						  Biología
				  \item
					  Finanzas
				  \item
					  Física
				  \item
					  Química
				\end{itemize}
			\end{overlayarea}
		\column{.75\textwidth}
			\begin{overlayarea}{\textwidth}{\textheight}
				\begin{exampleblock}{
						\only<1>{Lotka Volterra}
						\only<2>{Henston}
						\only<3>{Langevin}
						\only<4>{Brusselator}
					}
					 \only<1>{
						\begin{align*}
							dX_t &= (\lambda X_t - k X_t Y_t ) dt +\sigma X_t dW_t\\
							dY_t &= (k X_t Y_t -mY_t) dt
						\end{align*}
					}
					\only<2>{
						\begin{align*}
							dS_t &= \mu S_t dt + \sqrt{V_t}S_t
								\left(
									\sqrt{1- \rho^2}dW^{(1)}_t
									+ \rho dW^{(2)}_t
								\right)\\
							dV_t &=
								\kappa (\lambda - V_t)dt +
								\theta \sqrt{V_t} dW^{(2)}_t
						\end{align*}
					}
					\only<3>{
						\begin{equation*}
							dX_t = -(\nabla U)(X_t)dt + \sqrt{2\epsilon}dW_t
						\end{equation*}
					}
					\only<4>{
						\begin{align*}
							dX_t =& 
								\left[
									\delta
									-(\alpha + 1) X_t +
									Y_t X_t^2
								\right] dt
								+ g_1(X_t) dW_t^{(1)} \\
							dY_t =&
								\left[
									\alpha X_t +
									Y_t X_t^2
								\right] dt
								+ g_2(X_t) dW_t^{(2)} \\
						\end{align*}
					}
				\end{exampleblock}
			\end{overlayarea}
	\end{columns}
\end{frame}
%%%%%%%%%%%%%%%%%%%%%%%%%%%%%%%%%%%%%%%%%%%%%%%%%%%%%%%%%%%%%%%%%%%%%%%%%
